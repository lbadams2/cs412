\documentclass{article}
\usepackage[utf8]{inputenc}
\usepackage{amsmath}
\title{CS412 HW1}
\author{Liam Adams }
\date{September 20 2018}

\begin{document}

\maketitle

\section{Question 1}
Midterm vector $$\vec{m} = \begin{bmatrix} 75 & 78 & 84 & 86 & 88 & 90 & 95 & 96 & 96 & 99 \end{bmatrix}^T$$\\
Final vector $$\vec{f} = \begin{bmatrix} 77 & 80 & 80 & 85 & 88 & 88 & 90 & 95 & 99 & 100 \end{bmatrix}^T$$\\\\
a) Max Midterm = 99, Min Midterm = 75, Max Final = 100, Min Final = 77\\\\
b) The mean of the midterm grades is $$\mu_{mid}=\frac{1}{N}\sum\limits_{i=1}^{N} m_i=\frac{75+78+...+99}{10}=88.7$$\\
   The mode of the midterm grades is 96\\
   The median of the midterm grades is $\frac{88+90}{2}=89$\\\\
   The mean of the final grades is $$\mu_{final}=\frac{1}{N}\sum\limits_{i=1}^{N}f_i=\frac{77+80+...+100}{10}=88.2$$\\
   The mode of the final grades is 80 and 88\\
   The median of the final grades is $\frac{88+88}{2}=88$\\\\
c) The first quartile is the number for which at most $\frac{1}{4}$ of the data values are less than that number.  $\frac{10}{4}=2.5$ so the first quartile for a sample size of 10 would be at the 3rd member of the sample which is 84 for the midterm and 80 for the final\\
The third quartile is the number for which at most $\frac{3}{4}$ of the data values are less than that number.  $\frac{10x3}{4}=7.5$ so the third quartile for a sample size of 10 would be at the 8th member of the sample which is 96 for the midterm and 95 for the final\\
The inter-quartile range is the third quartile - the first quartile which is 96-84=12 for the midterm and 95-80=15 for the final\\\\
d) The sample variance is defined as $$s=\frac{1}{n-1}\sum\limits_{i=1}^{n}{(x_i-\bar{x})^2}$$. For $$\vec{m}$$ with n=10 and $$\bar{x}=88.7$$ we have $$s^2_{mid}=\frac{(75-88.7)^2 + (78-88.7)^2 + ... + (99-88.7)^2}{10-1}=65.122$$\\
For $$\vec{f}$$ with n=10 and $$\bar{x}=88.2$$ we have $$s^2_{final}=\frac{(77-88.2)^2 + (80-88.2)^2 + ... + (100-88.2)^2}{10-1}=63.956$$\\\\
The population variance is defined as $$\sigma^2=\frac{1}{N}\sum\limits_{i=1}^{N}{(x_i-\mu)^2}$$.  For $$\vec{m}$$ with N=10 and $$\mu=88.7$$ we have $$\sigma^2_{mid}=\frac{(75-88.7)^2 + (78-88.7)^2 + ... + (99-88.7)^2}{10}=58.610$$\\
For $$\vec{f}$$ with N=10 and $$\mu=88.2$$ we have $$\sigma^2_{final}=\frac{(77-88.2)^2 + (80-88.2)^2 + ... + (100-88.2)^2}{10}=57.560$$\\\\
e) The sample standard deviation s is defined as $$\sqrt{s^2}$$, so $$s_{mid}=\sqrt{65.122}=8.070$$ and $$s_{final}=\sqrt{63.956}=7.997$$\\
The population standard deviation is defined as $$\sqrt{\sigma^2}$$, $$\sigma_{mid}=\sqrt{58.610}=7.656$$ and $$\sigma_{final}=\sqrt{57.560}=7.587$$\\
\section{Question 2}
a) The min-max normalization is defined as $$x'_i=\frac{x_i-min_X}{max_X-min_X}$$\\
The min-max normalization for the midterm scores of students 1,2, and 3 is:\\
$$Student 1: x'_1=\frac{75-75}{99-75}=0$$\\
$$Student 2: x'_2=\frac{78-75}{99-75}=.125$$\\
$$Student 3: x'_3=\frac{84-75}{99-75}=.375$$\\
b) The population variance is defined as $$\sigma^2=\frac{1}{N}\sum\limits_{i=1}^{N}{(x_i-\mu)^2}$$. Plugging in the min-max normalized midterm scores gives $$\sigma^2_{mid}=\frac{(0-.571)^2 + (.125-.571)^2 + ... + (1-.571)^2}{10}=.102$$\\
c) The z-score normalization is defined as $$z_i=\frac{x_i-\mu}{\sigma}$$\\
The z-score normalization for the final scores of students 4,5, and 6 is:\\
$$Student 4: z_4=\frac{85-88.2}{7.587}=-.422$$\\
$$Student 5: z_5=\frac{88-88.2}{7.587}=-.026$$\\
$$Student 6: z_6=\frac{88-88.2}{7.587}=-.026$$\\
d) Using the same formula from part b we have $$\sigma^2_{final}=\frac{(-1.476-0)^2 + (-1.081-0)^2 + ... + (1.555-0)^2}{10}=1$$\\
\section{Question 3}
Midterm vector $$\vec{m} = \begin{bmatrix} 95 & 86 & 78 & 99 & 84 & 90 & 88 & 75 & 96 & 96 \end{bmatrix}^T$$\\
Final vector $$\vec{f} = \begin{bmatrix} 88 & 88 & 90 & 95 & 85 & 77 & 99 & 80 & 100 & 80 \end{bmatrix}^T$$\\\\
a) The population covariance is defined as $$E[\vec{m}\vec{f}]-E[\vec{m}]E[\vec{f}]$$ which is $$\frac{95\times88+86\times88+...+96\times80}{10}-88.7\times88.2=18.16$$\\
b) Pearson's correlation coefficient is defined as $$\frac{\sigma_{12}}{\sigma_1\sigma_2}$$ which is $$\frac{18.16}{7.656\times7.587}=.313$$
c) No because both the covariance and the correlation coefficient do not equal 0\\
d) Manhattan distance is defined as $$d(i,j)=|m_1-f_1|+|m_2-f_2|+...+|m_n-f_n|$$ which is $$|95-88|+|86-88|+...+|96-80|=75$$\\
Euclidean distance is defined as $$d(i,j)=\sqrt{|m_1-f_1|^2+|m_2-f_2|^2+...+|m_n-f_n|^2}$$ which is $$\sqrt{|95-88|^2+|86-88|^2+...+|96-80|^2}=28.302$$\\
Supremum distance is the maximum distance between any attribute of the vectors which is $$96-80=16$$\\
The cosine similarity is $$\frac{\vec{m}\cdot\vec{f}}{||\vec{m}||||\vec{f}||}$$ which is $$\frac{95\times88+86\times88+...+96\times80}{\sqrt{95^2+86^2+...+96^2}\times\sqrt{88^2+88^2+...80^2}}=.995$$
e) The Euclidean distance is the distance between the tips of the $$\vec{m}$$ and $$\vec{f}$$ vectors in 10 dimensional space.\\\\
f) No, because KL Divergence is used to measure the difference between 2 probability distributions over the same variable X.  But here we have to variables, M for the midterm scores and F for the final scores.  The Jaccard coefficient would also not be a good distance measure because it is used for assymetric binary variables and our variables are numeric.
\section{Question 4}
a) $$\chi^2=\sum\limits_{i=1}^{n}{\frac{(O_i-E_i)^2}{E_i}}$$ which is $$\frac{(200-\frac{280\times220}{3300})^2}{\frac{280\times220}{3300}}+\frac{(80-\frac{280\times3080}{3300})^2}{\frac{280\times3080}{3300}}+\frac{(20-\frac{3020\times220}{3300})^2}{\frac{3020\times220}{3300}}+\frac{(3000-\frac{3020\times3080}{3300})^2}{\frac{3020\times3080}{3300}}=2062.333$$\\
b) Yes, the degree of freedom is 1 and 2062.333 is greater than every value in the row for DF=1, so we can say purchasing beer and purchasing diapers are correlated with .001 confidence.\\\\
c) $$p_0=\frac{200}{3300}=.061$$
$$p_1=\frac{80+20}{3300}=.030$$
$$p_2=\frac{3000}{3300}=.909$$
$$\vec{p} = \begin{bmatrix} .061 & .030 & .909 \end{bmatrix}^T$$\\
d) $$D_{KL}(p(x)||q(x))=\sum\nolimits_{x \in X}{p(x)\ln{\frac{p(x)}{q(x)}}}$$ when q(x) is used to approximate p(x).  Using p(x) to approximate q(x) is the following
$$D_{KL}(q(x)||p(x))=\sum\nolimits_{x \in X}{q(x)\ln{\frac{q(x)}{p(x)}}}$$
$$.5\times\ln{\frac{.5}{.061}}+.3\times\ln{\frac{.3}{.030}}+.2\times\ln{\frac{.2}{.909}}=1.440$$
\end{document}
